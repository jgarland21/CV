%-------------------------
% Resume in Latex
% Author : Aras Gungore
% License : MIT
%------------------------

\documentclass[letterpaper,11pt]{article}

\usepackage{latexsym}
\usepackage[backend=biber,
minnames=5,
maxnames=5,
sorting=none]{biblatex}
\usepackage[empty]{fullpage}
\usepackage{titlesec}
\usepackage{marvosym}
\usepackage[usenames,dvipsnames]{color}
\usepackage{verbatim}
\usepackage{enumitem}
\usepackage[hidelinks,pdfnewwindow]{hyperref}
\usepackage{fancyhdr}
\usepackage[english]{babel}
\usepackage{tabularx}
\usepackage{hyphenat}
\usepackage{fontawesome}
\input{glyphtounicode}
\usepackage{orcidlink}
\usepackage{tikz}

%---------- FONT OPTIONS ----------
% sans-serif
% \usepackage[sfdefault]{FiraSans}
% \usepackage[sfdefault]{roboto}
% \usepackage[sfdefault]{noto-sans}
% \usepackage[default]{sourcesanspro}

% serif\
% \usepackage{CormorantGaramond}
% \usepackage{charter}

\addbibresource{main.bib}

\pagestyle{fancy}
\fancyhf{} % clear all header and footer fields
\fancyfoot{}
\renewcommand{\headrulewidth}{0pt}
\renewcommand{\footrulewidth}{0pt}

% Adjust margins
\addtolength{\oddsidemargin}{-0.5in}
\addtolength{\evensidemargin}{-0.5in}
\addtolength{\textwidth}{1in}
\addtolength{\topmargin}{-.5in}
\addtolength{\textheight}{1.0in}

\urlstyle{same}

\raggedbottom
\raggedright
\setlength{\tabcolsep}{0in}

% Sections formatting
\titleformat{\section}{
  \vspace{-4pt}\scshape\raggedright\large
}{}{0em}{}[\color{black}\titlerule \vspace{-5pt}]

% Ensure that generate pdf is machine readable/ATS parsable
\pdfgentounicode=1

%-------------------------
% Custom commands

%\tikzset{every picture/.append style={scale=1.4}}

\providecommand\mnras{\rm{MNRAS}}

\renewbibmacro{in:}{}

\renewcommand*{\bibfont}{\small}

\renewcommand*{\mkbibnamegiven}[1]{%
  \ifitemannotation{highlight}
    {\textbf{#1}}
    {#1}}

\renewcommand*{\mkbibnamefamily}[1]{%
  \ifitemannotation{highlight}
    {\textbf{#1}}
    {#1}}

\newcommand{\resumeItem}[1]{
  \item\small{
    {#1 \vspace{-2pt}}
  }
}


\newcommand{\resumeSubheading}[4]{
  \vspace{-2pt}\item
    \begin{tabular*}{0.97\textwidth}[t]{l@{\extracolsep{\fill}}r}
      \textbf{#1} & #2 \\
      \textit{\small#3} & \textit{\small #4} \\
    \end{tabular*}\vspace{-7pt}
}


\newcommand{\resumeSubSubheading}[2]{
    \vspace{-2pt}\item
    \begin{tabular*}{0.97\textwidth}{l@{\extracolsep{\fill}}r}
      \textit{\small#1} & \textit{\small #2} \\
    \end{tabular*}\vspace{-7pt}
}


\newcommand{\resumeEducationHeading}[4]{
  \vspace{-2pt}\item
    \begin{tabular*}{0.97\textwidth}[t]{l@{\extracolsep{\fill}}r}
      \textbf{#1} & #2 \\
      \textit{\small#3} & \textit{\small #4} \\
    \end{tabular*}\vspace{-5pt}
}


\newcommand{\resumeProjectHeading}[2]{
    \vspace{-2pt}\item
    \begin{tabular*}{0.97\textwidth}{l@{\extracolsep{\fill}}r}
      \small#1 & #2 \\
    \end{tabular*}\vspace{-7pt}
}


\newcommand{\resumeOrganizationHeading}[4]{
  \vspace{-2pt}\item
    \begin{tabular*}{0.97\textwidth}[t]{l@{\extracolsep{\fill}}r}
      \textbf{#1} & \textit{\small #2} \\
      \textit{\small#3}
    \end{tabular*}\vspace{-7pt}
}

\newcommand{\resumeSubItem}[1]{\resumeItem{#1}\vspace{-4pt}}

\renewcommand\labelitemii{$\vcenter{\hbox{\tiny$\bullet$}}$}

\newcommand{\resumeSubHeadingListStart}{\begin{itemize}[leftmargin=0.15in, label={}]}
\newcommand{\resumeSubHeadingListEnd}{\end{itemize}}
\newcommand{\resumeItemListStart}{\begin{itemize}}
\newcommand{\resumeItemListEnd}{\end{itemize}\vspace{-5pt}}

%-------------------------------------------
%%%%%%  RESUME STARTS HERE  %%%%%%%%%%%%%%%%%%%%%%%%%%%%


\begin{document}

%---------- HEADING ----------

\begin{center}
    \textbf{\Huge \scshape James T. Garland} \\ \vspace{3pt}
    \small
    \faMobile \hspace{.5pt} \href{tel:9176557955}{+1 (917) 655-7955}
    $|$
    \faAt \hspace{.5pt} \href{mailto:james.garland@mail.utoronto.ca}{james.garland@mail.utoronto.ca}
    $|$
    \faGlobe \hspace{.5pt} \href{https://www.jamesgarland.net}{jamesgarland.net}
    $|$
    \orcidlink{0000-0003-2922-1416}
    % \\
    % \faMapMarker \hspace{.5pt} {1120 Park Ave., New York, NY 10128, USA}
\end{center}



%----------- EDUCATION -----------

\section{Education}
  \vspace{3pt}
  \resumeSubHeadingListStart
    
    \resumeEducationHeading
      {Haverford College}{Haverford, PA, USA}
      {B.Sc. in Physics and Astronomy}{Sep 2018 \textbf{--} May 2022}
      %{Minor Degree in Computer Engineering;   \textbf{GPA: 3.89/4.00}}{Oct 2020 \textbf{--} Jun 2023 (Expected)}
        \resumeItemListStart
            \resumeItem{Thesis: \textit{The Interplay of Tides, Bars, and Star Formation in Disk Galaxies}}
            \resumeItem{Departmental High Honors}
        \resumeItemListEnd
  \resumeSubHeadingListEnd
    
%----------- RESEARCH EXPERIENCE -----------

\section{Research Experience}
  \vspace{3pt}
  \resumeSubHeadingListStart
  
    \resumeSubheading
      {American Museum of Natural History}{New York City, NY, USA}
      {Research Assistant}{Jul 2022 \textbf{--} Jul 2023}
        \resumeItemListStart
            \resumeItem{Worked with Dr. Michael Shara and the \href{https://condorarraytelescope.org/}{\color{blue}Condor Array Telescope} collaboration on identifying extragalactic novae and low-surface-brightness nova remnants in multi-epoch broad- and narrow-band images.}
            \resumeItem{Developed an automated source detection, photometry, and classification pipeline to identify transients in multi-epoch images and line-emission sources in multi-wavelength images.}
            \resumeItem{Used image correlation techniques between Condor and archival data to quantify the expansion of the nova shell around Z Cam.}
            \resumeItem{Planned observations of nova remnants for two SALT DDT proposals and analyzed the resulting RSS longslit spectra.}
            \resumeItem{Collaborated on an HST Cycle 31 proposal to measure extragalactic nova rates.}
        \resumeItemListEnd
    
    \resumeSubheading
      {Haverford College}{Haverford, PA, USA}
      {Undergraduate Research Intern (Multiple Internships)}{Sep 2019 \textbf{--} May 2022}
        \resumeItemListStart
            \resumeItem{Worked in the research groups of Professors Karen Masters (four semesters) and Daniel Grin (two semesters), including two 10-week summer internships funded by KINSC fellowship grants.}
            \resumeItem{Studied the tidal triggering/destruction of bars in galaxies using Galaxy Zoo citizen science data. Incorporated morphological, environmental, and star formation measures to inform a more comprehensive perspective on galaxy evolution.}
            \resumeItem{Developed mock survey code to generate and observe populations of dark matter haloes under different cosmologies with simulated HI surveys. A manuscript on the limits of galaxy surveys for constraining axion dark matter models is currently in preparation.}
            \resumeItem{Presented posters and talks at multiple national, local, collaboration, and collegiate consortium meetings.}
        \resumeItemListEnd
    
  \resumeSubHeadingListEnd



%----------- WORK EXPERIENCE -----------

\section{Work Experience}
  \vspace{3pt}
  \resumeSubHeadingListStart

    \resumeSubheading
      {Haverford College Public Observing}{Haverford, PA, USA}
      {Co-Head (2021-2022), Volunteer}{2018 \textbf{--} 2022}
        \resumeItemListStart
            \resumeItem{Organized and ran public events for college and local communities.}
            \resumeItem{Conducted observing sessions, talks, Q\&A sessions, and observatory tours.}
            \resumeItem{Operated and trained students in the use of 8", 12", and 16" telescopes at Strawbridge Observatory.}
        \resumeItemListEnd
    
    \resumeSubheading
      {Haverford College}{Haverford, PA, USA}
      {Teaching Assistant}{Jan 2021 \textbf{--} May 2021}
        \resumeItemListStart
            \resumeItem{Held office hours, assisted with observing sessions, and graded coursework for Astronomy 101.}
        \resumeItemListEnd
    
    % \resumeSubheading
      % {Ankara Metropolitan Municipality}{Ankara, Turkey}
      % {SCADA Engineering Intern}{Aug 2021 \textbf{--} Sep 2021}
        % \resumeItemListStart
            % \resumeItem{Designed GSM/GPRS-based electrical control panels that are connected to local water pump automation systems. Pump station panels use digital output data received from the SCADA control center via RF transmission to control valves and pumps. Tank station panels are charged from the PV system and refill water tanks by signaling the pump station panel when the float switch is activated.}
            % \resumeItem{Implemented motor control circuits by reading their PLC ladder diagrams and analyzed the EPLAN project documentation, HMI, and hardware components of an RTU panel.}
        % \resumeItemListEnd

    % \vspace{15pt}
    % \resumeSubheading
    %   {Meteksan Defense Industry Inc.}{Ankara, Turkey}
    %   {Analog Design Engineering Intern}{Jul 2021 \textbf{--} Aug 2021}
    %     \resumeItemListStart
    %         \resumeItem{Designed numerous analog circuits such as voltage-mode controlled buck converter, phase-shifted full-bridge isolated DC-DC converter, and EMI filters with LTspice. Integrated these circuits and implemented a 320 W power distribution unit to be used in a radar system's power circuit board.}
    %         \resumeItem{Researched real-world compatible electronic components to be used in such circuits including GaNFETs, high-side gate drivers, and Schottky diodes.}
    %         \resumeItem{Assembled PCBs of both common and differential mode filters and used VNA Bode 100 to measure the cut-off frequencies.}
    %     \resumeItemListEnd
    
  \resumeSubHeadingListEnd



%----------- AWARDS & HONORS -----------

\section{Awards \& Honors}
  \vspace{2pt}
  \resumeSubHeadingListStart
    \small{\item{
        \textbf{Louis B. Green Prize in Physics and Astronomy (2022):}{ Awarded to the graduating students who go above and beyond in their contributions to research and/or department culture and events.} \\ \vspace{3pt}
    
        \textbf{Chambliss Astronomy Achievement Award, Undergraduate Honorable Mention (2022):}{ Awarded for poster presented at the 240th meeting of the American Astronomical Society.} \\ \vspace{3pt}

        \textbf{KINSC Scientific Imaging Contest, Honorable Mention (2022):}{ Awarded for student-submitted images from experiments or simulations that are scientifically intriguing as well as aesthetically pleasing. \href{https://www.haverford.edu/college-communications/news/2022-kinsc-scientific-imaging-contest-winners-announced}{(\emph{\color{blue}Submission})}}
    }}
  \resumeSubHeadingListEnd

%----------- PUBLICATIONS -----------

\section{Publications}
    \vspace{3pt}

    \nocite{*}
    \printbibliography[heading=none]

    % \resumeSubHeadingListStart
      
    %   \resumeProjectHeading
    %     {\textit{Exploring the Roles of Galaxy Star Formation and Environment in the Tidal Triggering of Bars} $|$ \emph{\href{https://aas240-aas.ipostersessions.com/default.aspx?s=8F-A1-98-AC-04-5C-54-D0-F4-50-96-29-2A-A2-5C-49}{\color{blue}Poster}}}{}
    %       \resumeItemListStart
    %         \resumeItem{A C project which implements a variety of image processing operations that manipulate the size, filter, brightness, contrast, saturation, and other properties of PPM images from scratch.}
    %         \resumeItem{Added recursive fractal generation functions to model popular fractals including Mandelbrot set, Julia set, Koch curve, Barnsley fern, and Sierpinski triangle in PPM format.}
    %       \resumeItemListEnd
      
    %   \resumeProjectHeading
    %     {\textbf{Chess Bot} $|$ \emph{\href{https://github.com/arasgungore/chess-bot}{\color{blue}GitHub}}}{}
    %       \resumeItemListStart
    %         \resumeItem{A C++ project in which you can play chess against an AI with a specified decision tree depth that uses alpha-beta pruning algorithm to predict the optimal move.}
    %         \resumeItem{Aside from basic moves, this mini chess engine also implements chess rules such as castling, en passant, fifty-move rule, threefold repetition, and pawn promotion.}
    %       \resumeItemListEnd
      
    %   \resumeProjectHeading
    %     {\textbf{Rocket Flight Simulator} $|$ \emph{\href{https://github.com/arasgungore/rocket-flight-simulator}{\color{blue}GitHub}}}{}
    %       \resumeItemListStart
    %         \resumeItem{A Simulink project which can accurately simulate the motion of a flying rocket in one-dimensional space.}
    %         \resumeItem{The project implements the forces acting on a rocket which are drag, weight, and thrust as subsystems that take time-variant parameters into consideration such as distance from the center of Earth, mass and velocity of the rocket, and air density at different layers of Earth's atmosphere.}
    %       \resumeItemListEnd
      
    %   \resumeProjectHeading
    %     {\textbf{Netlist Solver} $|$ \emph{\href{https://github.com/arasgungore/netlist-solver}{\color{blue}GitHub}}}{}
    %       \resumeItemListStart
    %         \resumeItem{A MATLAB project that uses modified nodal analysis (MNA) algorithm to calculate the node voltages of any analog circuit without dependent sources given in netlist format.}
    %         \resumeItem{Added a module that sweeps the resistance of a load resistor, plots power dissipation as a function of load resistance, and finds the resistance value corresponding to maximum power.}
    %       \resumeItemListEnd
      
    %   \resumeProjectHeading
    %     {\textbf{CMPE 250 Projects} $|$ \emph{\href{https://github.com/arasgungore/CMPE250-projects}{\color{blue}GitHub}}}{}
    %       \resumeItemListStart
    %         \resumeItem{Five Java projects assigned for the Data Structures and Algorithms (CMPE 250) course in the Fall 2021-22 semester.}
    %         \resumeItem{These projects apply DS\&A concepts such as discrete-event simulation (DES) using priority queues, Dijkstra's shortest path algorithm, Prim's algorithm to find the minimum spanning tree (MST), Dinic's algorithm for maximum flow problems, and weighted job scheduling with dynamic programming to real-world problems.}
    %       \resumeItemListEnd
      
    % \resumeSubHeadingListEnd

%----------- TALKS & PRESENTATIONS -----------

\section{Talks \& Presentations}
  \vspace{2pt}
  \resumeSubHeadingListStart
    \small{\item{
        {Talk, Galaxy Zoo 15th anniversary telecon, Jul 2022.} \textit{Exploring the Roles of Galaxy Star Formation and Environment in the Tidal Triggering of Bars.} (\href{https://docs.google.com/presentation/d/15VxjLvjgd2ll7C_z6z5aoNgNm0caDdDY4YJobk2M9Ek/edit?usp=sharing}{\emph{\color{blue}Slides}}) \\ \vspace{3pt}
    
        {Poster, AAS 240th meeting, Jun 2022.} \textit{Exploring the Roles of Galaxy Star Formation and Environment in the Tidal Triggering of Bars.} (\href{https://aas240-aas.ipostersessions.com/default.aspx?s=8F-A1-98-AC-04-5C-54-D0-F4-50-96-29-2A-A2-5C-49}{\emph{\color{blue}Poster}}) \\ \vspace{3pt}

        {Talk, Galaxy Zoo biweekly telecon, Dec 2021.} \textit{Exploring the Roles of Galaxy Star Formation and Environment in the Tidal Triggering of Bars.} (\href{https://docs.google.com/presentation/d/1jzij3Ceulp25-IIE3rHtpkHmbfgnpxNTJdtfsHkyeMA/edit?usp=sharing}{\emph{\color{blue}Slides}}) \\ \vspace{3pt}

        {Talk, 32nd annual Keck Northeast Astronomy Consortium meeting, Sep 2021.} \textit{Exploring the Roles of Galaxy Star Formation and Environment in the Tidal Triggering of Bars.} (\href{https://sites.google.com/haverford.edu/knac2021/presentations?authuser=0}{\emph{\color{blue}Abstract}}, \href{https://youtu.be/3Ki3iiklxO4}{\emph{\color{blue}Recording}}) \\ \vspace{3pt}

        {Poster, Haverford KINSC Undergraduate Science Research Symposium, Sep 2021.} \textit{Exploring the Roles of Galaxy Star Formation and Environment in the Tidal Triggering of Bars.} \\ \vspace{3pt}

        {Poster, AAS 237th meeting, Jan 2021.} \textit{Can HI Observations of Low-Mass Galaxies Test Ultra-Light Axion Dark Matter?} (\href{https://baas.aas.org/pub/2021n1i149p02/release/1}{\emph{\color{blue}Abstract}}, \href{https://aas237-aas.ipostersessions.com/default.aspx?s=83-87-6E-2B-3B-93-FC-40-63-5C-44-DF-0D-54-65-42}{\emph{\color{blue}Poster}}) \\ \vspace{3pt}

        {Poster, Haverford KINSC Undergraduate Science Research Symposium, Oct 2020.} \textit{Can We Test Axion Dark Matter Models With Galaxy Surveys?} \\ \vspace{3pt}

        {Poster, 31st annual Keck Northeast Astronomy Consortium meeting, Oct 2020.} \textit{Can HI Observations of Low-Mass Galaxies Test Ultra-Light Axion Dark Matter?} (\href{https://sites.google.com/view/knac2020/posters?authuser=0}{\emph{\color{blue}Abstract}}) \\ \vspace{3pt}

        {``Lightning talk'', 2020 SDSS-IV/V Collaboration Meeting, Jun 2020.} \textit{Can HI Observations of Low-Mass Galaxies Test Ultra-Light Axion Dark Matter?}
    }}
  \resumeSubHeadingListEnd

%----------- SKILLS -----------

\section{Skills}
  \vspace{2pt}
  \resumeSubHeadingListStart
    \small{\item{
        \textbf{Technical:}{ Python, Data Reduction and Analysis, Data Visualization, Optical Telescope Operation, CCD Image Reduction, Observation Planning, Astrophotography} \\ \vspace{3pt}
        
        \textbf{Communications:}{ Proposal Writing, Scientific Writing, Science Communication, Public Outreach}
    }}
  \resumeSubHeadingListEnd



%----------- RELEVANT COURSEWORK -----------

\section{Relevant Coursework}
  \vspace{2pt}
  \resumeSubHeadingListStart
    \small{\item{
        \textbf{Astronomy (Undergraduate):}{ Intro Astrophysics, Observational Astronomy, Multi-Wavelength Astronomy, Galactic Dynamics \& Mechanics (mixed undergraduate \& graduate), Gravitational Waves, Extragalactic Data Science} \\ \vspace{3pt}

        \textbf{Physics (Undergraduate):}{ Fundamental Physics I-II, Waves and Optics, Advanced Quantum Mechanics, Advanced Classical Mechanics, Advanced Electromagnetism} \\ \vspace{3pt}

        \textbf{Misc. (Undergraduate):}{ Multivariable Calculus, Linear Algebra, History of Science}
    }}
  \resumeSubHeadingListEnd



%----------- CERTIFICATES -----------

% \section{Certificates}
  % \resumeSubHeadingListStart
    
    % \resumeOrganizationHeading
      % {Procter \& Gamble VIA Certificate Program}{Feb 2022}{Business Skills, Data and Digital Skills, Project Management and Personal Productivity}
    
  % \resumeSubHeadingListEnd



%----------- ORGANIZATIONS -----------

% \section{Organizations}
  % \resumeSubHeadingListStart
    
    % \resumeOrganizationHeading
      % {Institute of Electrical and Electronics Engineers (IEEE)}{Feb 2022 -- Present}{Student Member}
    
  % \resumeSubHeadingListEnd



%----------- HOBBIES -----------

% \section{Hobbies}
  % \resumeSubHeadingListStart
    % \small{\item{Swimming, Fitness, Eight-ball}}
  % \resumeSubHeadingListEnd



%----------- REFERENCES -----------

% \section{References}
  % \resumeSubHeadingListStart
    
  % \resumeSubHeadingListEnd



%-------------------------------------------
\end{document}
